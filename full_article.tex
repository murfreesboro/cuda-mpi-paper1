\documentclass[num-refs]{wiley-article}

\usepackage{graphicx}
\usepackage[space]{grffile}
\usepackage{latexsym}
\usepackage{textcomp}
\usepackage{longtable}
\usepackage{tabulary}
\usepackage{booktabs,array,multirow}
\usepackage{amsfonts,amsmath,amssymb}
\usepackage{natbib}
\usepackage{url}
\usepackage{hyperref}
\hypersetup{colorlinks=false,pdfborder={0 0 0}}
\usepackage{etoolbox}
\makeatletter
%\patchcmd\@combinedblfloats{\box\@outputbox}{\unvbox\@outputbox}{}{%
%  \errmessage{\noexpand\@combinedblfloats could not be patched}%
%}%
\makeatother
% You can conditionalize code for latexml or normal latex using this.
\newif\iflatexml\latexmlfalse
\AtBeginDocument{\DeclareGraphicsExtensions{.pdf,.PDF,.eps,.EPS,.png,.PNG,.tif,.TIF,.jpg,.JPG,.jpeg,.JPEG}}

\usepackage[utf8]{inputenc}
\usepackage[english]{babel}

% Add any additional LaTeX packages and macros here
\usepackage{siunitx}

%---

\iflatexml
% Add any LateXML specific commands here

%---

\else
% The commands below will only change the exported PDF. Edit or remove as needed

%\paperfield{Field of the paper}
%\abbrevs{ABC, a black cat; DEF, doesn't ever fret; GHI, goes home immediately.}
\corraddress{5004 General Eisenhower Dr, Murfreesboro, TN, 37129, USA}
\corremail{fhilosophierr@gmail.com}
\presentadd{Department, Institution, City, State or Province, Postal Code, Country}
%\fundinginfo{Funder One, Funder One Department, Grant/Award Number: 123456, 123457 and 123458; Funder Two, Funder Two Department, Grant/Award Number: 123459}
\fi

%---

%\papertype{Original Article}

\title{}


\author[1]{Fenglai Liu}

\affil[1]{Authorea}



\runningauthor{Alberto Pepe}

\begin{document}

\maketitle
\selectlanguage{english}
\begin{abstract}
This is a LaTeX template designed for use by the \textbf{International Journal of Quantum Chemistry}. Please consult the journal's author guidelines in order to confirm that your manuscript complies with the journal's requirements. When you are ready to submit your manuscript, download/export it in LaTeX or Word format and submit your document at \href{https://mc.manuscriptcentral.com/qua}{https://mc.manuscriptcentral.com/qua}. Please replace this text with your abstract.

\textbf{Keywords} --- keyword 1, \emph{keyword 2}, keyword 3, keyword 4, keyword 5, keyword 6, keyword 7.%
\end{abstract}%

%%%%%%%%%%%%%%%%%%%%%%%%%%%%%%%%%%%%%%%%%%%%%%%%%%%%%%%%%%%
% mathematical details
%
% flow chart to do the integrals
%
%%%%%%%%%%%%%%%%%%%%%%%%%%%%%%%%%%%%%%%%%%%%%%%%%%%%%%%%%%%


\section{Introduction of four-centers integrals for Coulomb and exchange}


 
In quantum chemistry the basis functions are fundamental to a molecular QM calculation. Here
in this our discussion is focused on the Cartesian Gaussian form of functions: 
\begin{equation}\label{work:1}
         \chi = x^{i}y^{j}z^{k}e^{-\alpha r^{2}} \quad \text{.}
\end{equation}
In practice a basis function $\phi$ is generally a linear combination of Cartesian 
Gaussian functions:
\begin{equation}\label{work:2}
         \phi = \sum_{n}d_{n}\chi_{n} \quad \text{,}
\end{equation}
where the coefficient $d_{n}$ is an pre-optimized constant. Upon the Gaussian form basis functions, The fundamental Coulomb and HF exchange energies can be expressed as:
\begin{align}
 \label{eq:1}
         E_{Coulomb}  &= \frac{1}{2}\sum_{\mu\nu}P_{\mu\nu}
 \sum_{\lambda\eta}P_{\lambda\eta}\int dr dr^{'}\frac{\phi_{\mu}(r)\phi_{\nu}(r)
 \phi_{\lambda}(r^{'})\phi_{\eta}(r^{'})} {|r - r^{'}|} \nonumber \\
         E_{Exchange} &= -\frac{1}{2}\sum_{\mu\nu}P_{\mu\nu}
 \sum_{\lambda\eta}P_{\lambda\eta}\int dr dr^{'}\frac{\phi_{\mu}(r)\phi_{\lambda}(r)
 \phi_{\nu}(r^{'})\phi_{\eta}(r^{'})} {|r - r^{'}|} \quad \text{.}
\end{align}
$r$ and $r^{'}$ characterize the positions for the two electrons.
$\phi$ is a basis function, and $\mu$, $\nu$, $\lambda$ and $\eta$ are the
indexes for the basis functions. $P_{\mu\nu}$ is the density
matrix.

The computation the Coulomb and HF exchange becomes the most time-consuming
part in QM applications is because the electron repulsion integral(ERI) in Eq.
\ref{eq:1} involves four indexes, so theoretically it generally scales as
$N^{4}$ (N is the size of basis functions).  Furthermore since each basis
function is also a contraction of Gaussian functions, such ERI calculation
actually involves eight loops. 

In practice the computation of ERI is usually performed on shell-quartet,
where all of ERIs coming from the same four-shell set are calculated together
so that to save the computational cost. The shell-quartet is usually abbreviated 
as $(ab|cd)$, where $a$, $b$, $c$ and $d$ etc. represent the general shells data on
the positions of $\mu$, $\nu$, $\lambda$ and $\eta$ in Eq. \ref{eq:1}.
In particular we usually use ``0'' to represent the S shell($L=0$). For example,
$(e0|f0)$ means the shells data on $\nu$ and $\eta$ positions are S type of shells.

The most efficient algorithm for directly performing ERI calculation is to use
recurrence relations\cite{gill1994molecular}, and the HGP algorithm\cite{HGP}
provides the most efficient recurrence relations for the general ERI
derivation.  The calculation of ERI in the HGP algorithm can be general divided
into two steps.  The first step is to perform vertical recurrence
relations(VRR) which transform the directly computed auxiliary integrals
$(00|00)^{m}$ into the shell-quartets in the form of $(e0|f0)$.  The second step is
to conduct horizontal recurrence relations(HRR), and in this step the intermediate
shell-quartets $(e0|f0)$ will be finally transformed to result shell-quartet
$(ab|cd)$.

Generally the cost of ERI calculation(approximated FLOPS cost) scales at
$(L+1.39)^{8}$\cite{shao2000improved}.  Due to such rapid increase of the cost
the implementation of ERI on the MIC platform is needed to be divided into two
parts. The first part is the implementation for low angular momentum basis
functions including those in S, P and D types.  The second part is for basis
functions with angular momentum higher than D.



%%%%%%%%%%%%%%%%%%%%%%%%%%%%%%%%%%%%%%%%%%%%%%%%%%%%%%%%%%%
% GPU implementation 
%
% with integral or shell quartet?
%
% general strategy (TLP or ILP)? How we really do it?
% choose thread/block size and unrolling.
%
% shared memory bank conflict
%
% no use of local storage, maximumly using the register/L1
%
% how many shell quartets averagely computed per second
%
%%%%%%%%%%%%%%%%%%%%%%%%%%%%%%%%%%%%%%%%%%%%%%%%%%%%%%%%%%%


\section{GPU implementation of four-center integrals}

For the ERIs in \ref{eq:1} the traditional way on CPU is to compute all of integrals together in the same shell quartet. For example the shell quartet $(PP|PP)$ is composed by four $P$ shells and it consists of 81 ERIs. The benefit to compute ERIs through the shell quartet is to save FLOPS since different ERIs in the same shell quartet may share the intermediate results during the integral computation. For GPU implementation because of the limitted resource of register and L1 cache, earier implementation of four-center integrals were generally computed based on ERI (ref), however recently as the integrals implementation goes to higher momentum such as F etc. the available GPU implementations are more based on shell quartets(ref). From the recent development path for GPU, the computing unit has larger resources as we discussed above; hence in the long run we think the shell quartet GPU implementation will probably become the mainstream for future GPU computation in quantum chemistry. In this paper our GPU implementation are also based on shell quartets (ref), and our current implementation only involves S and P shells.

Thread level parallelism (TLP) and instruction level parallelism (ILP) are two common strategies for GPU implementation. TLP concentrates on hiding the latency by spawning large number of threads. ILP focuses on improving the instruction set utilization rate, and one of the useful techanque for ILP is the loop unrolling. We believe a good GPU implementation for computing the ERIs come from the compromise of the TLP and ILP. Take the the GPU implementation on cuda as an example, for the shell quartets involving the low angular momentum S and P it's possible to put several shell quartets together inside one kernel function, therefore to use the loop
unrolling increases the ILP, and also enhances the global memory loading efficiency. However with the larger number of register being used the number of warps per block is getting smaller(less threads per block), hence the high lentency may not be able to cover dynamically. Is there a compromise point to achieve the best optimization? To explore the possibility we use $(SS|SS)$ integral with one contraction degree as an example. 

\begin{table}
\caption{{This is a table. Tables should be self-contained and complement, but not duplicate, information contained in the text. They should be not be provided as images. Legends should be concise but comprehensive -- the table, legend and footnotes must be understandable without reference to the text.}}
\begin{tabular}{lccrr}
\hline
\textbf{Variables} & \textbf{JKL\footnote{Just Keep Laughing} ($\boldsymbol{n=30}$)} & \textbf{Control ($\boldsymbol{n=40}$)} & \textbf{MN\footnote{Merry Nose}} & \textbf{$\boldsymbol t$ (68)}\\
Age at testing & 38 & 58 & 504.48 & 58 ms\\
Age at testing & 38 & 58 & 504.48 & 58 ms\\
Age at testing & 38 & 58 & 504.48 & 58 ms\\
Age at testing & 38 & 58 & 504.48 & 58 ms\\
Age at testing & 38 & 58 & 504.48 & 58 ms\\
Age at testing & 38 & 58 & 504.48 & 58 ms\\
\hline  
\end{tabular}
\end{table}


\subsection{Second Level Heading}
If data, scripts or other artefacts used to generate the analyses presented in the article are available via a publicly available data repository, please include a reference to the location of the material within the article.

This is an equation, numbered

\begin{equation}
\label{eqn:some}
\int_0^{+\infty}e^{-x^2}dx=\frac{\sqrt{\pi}}{2}
\end{equation}

And one that is not numbered

\begin{equation*}
e^{i\pi}=-1
\end{equation*}

\subsection{Adding Citations and a References List}

You can cite bibliographic entries easily in Authorea. For example, a couple of citations follow \cite{Cavalleri_2016}. By default citations are formatted according to the IJQC format, but you can change that in the Export Options \cite{Meskine_2019}

\subsubsection{Customizing the exported PDF}
You can add packages, create aliases and macros (newcommands), as well as specify corresponding address, corresponding email, funding information etc by clicking "Settings", "Edit Macros".

\paragraph{Fourth Level Heading}

\begin{quote}
The significant problems we have cannot be solved at the same level of thinking with which we created them.
\end{quote}


\subparagraph{Fifth level heading}
Measurements should be given in SI or SI-derived units.\selectlanguage{english}


Chemical substances should be referred to by the generic name only. Trade names should not be used. Drugs should be referred to by their generic names. If proprietary drugs have been used in the study, refer to these by their generic name, mentioning the proprietary name, and the name and location of the manufacturer, in parentheses.\selectlanguage{english}


%%%%%%%%%%%%%%%%%%%%%%%%%%%%%%%%%%%%%%%%%%%%%%%%%%%%%%%%
% 
% 
% 
% raw integral and digestion together
% 
% compare with GPU implementation
%
%%%%%%%%%%%%%%%%%%%%%%%%%%%%%%%%%%%%%%%%%%%%%%%%%%%%%%%%

\section{Mix of GPU and CPU implementations for a batch work}



%%%%%%%%%%%%%%%%%%%%%%%%%%%%%%%%%%%%%%%%%%%%%%%%%%%%%%%%
% 
% one ket to many bra shell pairs
% 
% raw integral and digestion together
% 
% compare with GPU implementation
%
%%%%%%%%%%%%%%%%%%%%%%%%%%%%%%%%%%%%%%%%%%%%%%%%%%%%%%%%

\section{CPU implementation of four-centers integrals}


\section*{Acknowledgements}
Acknowledgements should include contributions from anyone who does not meet the criteria for authorship (for example, to recognize contributions from people who provided technical help, collation of data, writing assistance, acquisition of funding, or a department chairperson who provided general support), as well as any funding or other support information.

\section*{Conflict of interest}
You may be asked to provide a conflict of interest statement during the submission process. Please check the journal's author guidelines for details on what to include in this section. Please ensure you liaise with all co-authors to confirm agreement with the final statement.

%\selectlanguage{english}
\bibliography{bibliography/ref.bib}

\end{document}

